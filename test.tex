
\section*{Data Science Ethics Checklist}

\subsection*{A. Data Collection}
\begin{itemize}
\item If there are human subjects, have they given informed consent, where subjects affirmatively opt-in and have a clear understanding of the data uses to which they consent?
\item Have we considered sources of bias that could be introduced during data collection and survey design and taken steps to mitigate those?
\item Have we considered ways to minimize exposure of personally identifiable information (PII) for example through anonymization or not collecting information that isn't relevant for analysis?
\item Have we considered ways to enable testing downstream results for biased outcomes (e.g., collecting data on protected group status like race or gender)?
\end{itemize}



\subsection*{B. Data Storage}
\begin{itemize}
\item Do we have a plan to protect and secure data (e.g., encryption at rest and in transit, access controls on internal users and third parties, access logs, and up-to-date software)?
\item Do we have a mechanism through which an individual can request their personal information be removed?
\item Is there a schedule or plan to delete the data after it is no longer needed?
\end{itemize}



\subsection*{C. Analysis}
\begin{itemize}
\item Have we sought to address blindspots in the analysis through engagement with relevant stakeholders (e.g., checking assumptions and discussing implications with affected communities and subject matter experts)?
\item Have we examined the data for possible sources of bias and taken steps to mitigate or address these biases (e.g., stereotype perpetuation, confirmation bias, imbalanced classes, or omitted confounding variables)?
\item Are our visualizations, summary statistics, and reports designed to honestly represent the underlying data?
\item Have we ensured that data with PII are not used or displayed unless necessary for the analysis?
\item Is the process of generating the analysis well documented and reproducible if we discover issues in the future?
\end{itemize}



\subsection*{D. Modeling}
\begin{itemize}
\item Have we ensured that the model does not rely on variables or proxies for variables that are unfairly discriminatory?
\item Have we tested model results for fairness with respect to different affected groups (e.g., tested for disparate error rates)?
\item Have we considered the effects of optimizing for our defined metrics and considered additional metrics?
\item Can we explain in understandable terms a decision the model made in cases where a justification is needed?
\item Have we communicated the shortcomings, limitations, and biases of the model to relevant stakeholders in ways that can be generally understood?
\end{itemize}



\subsection*{E. Deployment}
\begin{itemize}
\item Do we have a clear plan to monitor the model and its impacts after it is deployed (e.g., performance monitoring, regular audit of sample predictions, human review of high-stakes decisions, reviewing downstream impacts of errors or low-confidence decisions, testing for concept drift)?
\item Have we discussed with our organization a plan for response if users are harmed by the results (e.g., how does the data science team evaluate these cases and update analysis and models to prevent future harm)?
\item Is there a way to turn off or roll back the model in production if necessary?
\item Have we taken steps to identify and prevent unintended uses and abuse of the model and do we have a plan to monitor these once the model is deployed?
\end{itemize}

